\documentclass[11pt]{article}

\usepackage[margin=1in]{geometry}
\usepackage{graphicx}
\usepackage{booktabs}
\usepackage{longtable}
\usepackage{array}
\usepackage{float}
\usepackage{placeins}
\usepackage{caption}
\usepackage{subcaption}
\usepackage{amsmath}
\usepackage{amssymb}
\usepackage{siunitx}
\usepackage{microtype}
\usepackage[numbers]{natbib}
\usepackage[hidelinks]{hyperref}
\usepackage{setspace}

\onehalfspacing
\setlength{\parskip}{0.35em}
\setlength{\parindent}{0em}

\title{Leak-Safe Probabilistic Forecasting for the 2026 Wharton Hockey Competition}
\author{
  Sebastien Kawada, Dylan Holyoak, and Aidan Gildea\\
  Epoch Learn\\
  Los Angeles, USA
}
\date{\today}

\begin{document}
\maketitle

\begin{abstract}
We present a leak-safe, decision-theoretic forecasting system for the 2026 Wharton Data Science Competition hockey task.
The competition objective is pregame home-win probability estimation for 16 Round 1 matchups, where quality is judged by probabilistic scoring rather than postgame classification.
Using 1,312 historical games across 32 teams, we generate every feature vector from pregame state only, tune models with rolling cross-validation, and evaluate exactly once on an untouched temporal holdout.
The final stack combines Elo-style paired-comparison structure \cite{elo1978,bradley1952rank,glickman1999parameter}, supervised matchup learners \cite{cox1958regression,breiman2001,friedman2001greedy}, and calibration-aware selection \cite{platt1999,zadrozny2002transforming,niculescu2005predicting}.
On untouched holdout data, the deployed source (Elo-shrunk recent-objective selector) attains log loss 0.6720 versus 0.6759 for a home-rate baseline (delta -0.00387, -0.57\% relative), while also improving Brier score.
To characterize deployment risk, we report IID and moving-block bootstrap projections \cite{efron1979bootstrap,kunsch1989jackknife} and estimate a 70.7\%--74.5\% probability of beating baseline log loss on future samples.
\end{abstract}

\section{Introduction}
Sports prediction systems can look strong while relying on information that is unavailable at real prediction time.
This failure mode is common in competition settings where game summaries contain highly outcome-linked variables.
For the Wharton hockey task, the core challenge is to estimate pregame win probabilities, not to classify finished games from postgame box scores.
Because the output is probabilistic, model quality must be evaluated with proper scoring rules, not only thresholded classification accuracy \cite{brier1950,gneiting2007}.
Conceptually, the task is a dynamic paired-comparison problem \cite{bradley1952rank,glickman1999parameter} under temporal non-stationarity, where operational validity depends on strict chronology and calibration quality \cite{niculescu2005predicting}.

From a decision perspective, this is risk minimization under a proper score:
\[
\mathcal{R}(f)=\mathbb{E}\left[-Y\log f(X)-(1-Y)\log(1-f(X))\right],
\]
where \(f(X)\in(0,1)\) is the forecasted home-win probability.
Minimizing \(\mathcal{R}\) aligns model training and selection with the competition metric and penalizes overconfident mistakes superlinearly \cite{gneiting2007}.

This work makes four contributions:
\begin{samepage}
\begin{enumerate}
    \item A chronology-safe state-space feature engine where every game row is generated from pregame information only.
    \item A probabilistic model family spanning Elo, logistic regression \cite{cox1958regression}, random forests \cite{breiman2001}, and gradient boosting \cite{friedman2001greedy}.
    \item A recency-aware selection rule that balances global CV quality with late-fold robustness, then selects a deployable probability source.
    \item A reproducible research artifact set: deterministic runner, hash-based manifests, publication figures, and tables.
\end{enumerate}
\end{samepage}

\section{Task and Data}
The historical season dataset contains 1,312 games across 32 teams.
Each game is represented by multiple line-level records, which we aggregate to one game-level row before constructing pregame features.
Round 1 prediction requires 16 home-away matchups with no realized in-game statistics available at scoring time.
Formally, with chronological index \(t\in\{1,\dots,T\}\), each game contributes tuple
\[
g_t=(h_t,a_t,y_t,r_t),
\]
where \(h_t\) and \(a_t\) are teams, \(y_t\in\{0,1\}\) is the home-win indicator, and \(r_t\) is the postgame statistics bundle (goals, xG, shots, assists, penalties, and derivatives) used only to update future state.

\begin{table}[!htbp]
\centering
\caption{Experiment Summary and Key Counts}
\begin{tabular}{ll}
\toprule
metric & value \\
\midrule
Total games & 1312 \\
Development games & 1115 \\
Holdout games & 197 \\
Selected model & elo shrunk recent objective \\
Selected prediction source & elo shrunk home rate \\
Selected holdout log loss & 0.671998 \\
Baseline holdout log loss & 0.675868 \\
Delta log loss vs baseline & -0.003869 \\
\bottomrule
\end{tabular}

\label{tab:summary}
\end{table}

\section{Methodology}
\subsection{Leak-Safe Feature Construction}
Index games chronologically by \(t=1,\dots,T\), with home team \(h_t\), away team \(a_t\), and binary outcome \(y_t\in\{0,1\}\) (home win indicator).
Let \(s_t\) denote the full league state immediately before game \(t\).
Our feature map and state update are
\[
x_t=\phi(s_t,h_t,a_t),\qquad
p_t=f_\theta(x_t),\qquad
s_{t+1}=U(s_t,h_t,a_t,y_t,r_t),
\]
where \(r_t\) are game-realized statistics observed only after completion.
By construction, \(x_t\) is conditionally independent of \(y_t\) given \((s_t,h_t,a_t)\), which enforces temporal admissibility and blocks target leakage.
Equivalently, with information set \(\mathcal{F}_{t^-}\) immediately before game \(t\), we require \(x_t\in\mathcal{F}_{t^-}\) and \(y_t\notin\mathcal{F}_{t^-}\).
This filtration view makes leakage checks auditable: any feature not measurable in \(\mathcal{F}_{t^-}\) is invalid by definition.

Elo probabilities follow:
\[
p_{\text{home}}=\frac{1}{1+10^{-\frac{(R_{\text{home}}+H-R_{\text{away}})}{400}}}, \quad
R'_{\text{home}} = R_{\text{home}} + K(y_{\text{home}}-p_{\text{home}})
\]
where \(K\) is the update factor, \(H\) is home-advantage points, and \(y_{\text{home}} \in \{0,1\}\).
To improve early-season stability, we use an Elo-shrunk source:
\[
p_t^{\text{shrunk}}=\alpha\,p_t^{\text{elo}} + (1-\alpha)\,\pi_{\text{home}},
\]
where \(\pi_{\text{home}}\) is the development home-win prior and \(\alpha\in[0,1]\) is selected by CV.

\subsection{Model Stack}
For probabilistic selection we optimize strictly proper scores \cite{brier1950,gneiting2007}:
\[
\mathcal{L}_{\log} = -\frac{1}{N}\sum_{i=1}^{N}\left[y_i\log p_i + (1-y_i)\log(1-p_i)\right],\qquad
\mathcal{L}_{\text{Brier}} = \frac{1}{N}\sum_{i=1}^{N}(p_i-y_i)^2.
\]

We evaluate:
\begin{itemize}
    \item Elo-only probability baseline \cite{elo1978},
    \item Elo-shrunk probability source (Elo mixed with training home-rate prior),
    \item logistic regression on engineered matchup features \cite{cox1958regression},
    \item regularized random forest \cite{breiman2001},
    \item histogram gradient boosting \cite{friedman2001greedy},
    \item weighted convex blend,
    \item stacked logistic meta-model over component probabilities.
\end{itemize}
For blend candidates:
\[
p_t^{\text{blend}}=\sum_{j=1}^{J}w_j p_{t,j},\qquad
w_j\ge 0,\ \sum_j w_j=1.
\]
For stacked candidates:
\[
p_t^{\text{stack}}=\sigma(\beta_0+\beta^\top z_t),\quad
z_t=[p_t^{\text{elo}},p_t^{\text{elo-shrunk}},p_t^{\text{lr}},p_t^{\text{rf}},p_t^{\text{hgb}}],
\]
with optional post-hoc calibration \(p_t^{\text{cal}}=C(p_t)\) via Platt or isotonic mappings \cite{platt1999,zadrozny2002transforming,kull2017beta}.
Calibrated forecasts are interpreted through the reliability identity
\[
\mathbb{E}[Y\mid P=p]=p,
\]
which operationally links forecast probabilities to observed frequencies \cite{niculescu2005predicting}.
Calibration is retained only when it improves OOF log loss beyond a minimum threshold.

\subsection{Recent-Fold Robustness Selector}
Let \(\text{LL}_{m}^{\text{OOF}}\) and \(\text{LL}_{m}^{\text{recent}}\) be average and recency-weighted fold log losses for source \(m\).
For the recent component over the last \(K\) folds:
\[
\text{LL}_{m}^{\text{recent}}=\sum_{k=1}^{K}\omega_k \text{LL}_{m,k},\qquad
\omega_k=\frac{\rho^{K-k}}{\sum_{j=1}^{K}\rho^{K-j}},
\]
with \(\rho>1\) emphasizing the latest folds.
We score each source by
\[
J_m=\lambda\,\text{LL}_{m}^{\text{recent}} + (1-\lambda)\,\text{LL}_{m}^{\text{OOF}},
\]
then select the minimum-\(J_m\) source subject to robustness guardrails.
In practice, advanced blend/stack sources are accepted only if they beat Elo-family sources on both OOF and recent-fold criteria by explicit margins.
This design intentionally trades some average-CV optimality for reduced late-season fragility.

\begin{figure}[!htbp]
\centering
\includegraphics[width=\textwidth]{figures/fig_pipeline_overview.png}
\caption{Pipeline overview: chronology-safe feature generation, CV tuning, untouched holdout evaluation, and Round 1 scoring.}
\label{fig:pipeline}
\end{figure}

\section{Experimental Setup}
We use a strict temporal split:
first 85\% of games for model development and tuning, last 15\% as untouched holdout.
Primary metric is log loss, with Brier score as secondary probability metric.
Accuracy and AUC are reported for completeness.
All hyperparameter and source-selection decisions are made on development folds only.
Rolling time-series CV is used (rather than IID shuffling) to respect ordering assumptions and avoid optimistic leakage through temporal dependence \cite{bergmeir2018note}.
Model-family diversity is intentional: generalized linear, tree-bagged, and boosting learners have different bias-variance-calibration failure modes, and we preserve this diversity before final source selection \cite{breiman2001,friedman2001greedy,niculescu2005predicting}.
\FloatBarrier

\section{Results}
\subsection{Out-of-Fold Development Performance}
\begin{figure}[!htbp]
\centering
\includegraphics[width=0.85\textwidth]{figures/fig_cv_oof_model_comparison.png}
\caption{Cross-validated OOF model comparison on development data.}
\label{fig:oof_compare}
\end{figure}

\begin{table}[!htbp]
\centering
\caption{OOF Development Metrics}
\begin{tabular}{lrrrr}
\toprule
model & accuracy & log\_loss & brier & auc \\
\midrule
elo only & 0.5568 & 0.6868 & 0.2469 & 0.5636 \\
logistic regression & 0.5070 & 1.1259 & 0.3383 & 0.4953 \\
random forest & 0.5632 & 0.6914 & 0.2487 & 0.5560 \\
hist gradient boosting & 0.5308 & 0.9582 & 0.3182 & 0.5225 \\
elo shrunk home rate & 0.5546 & 0.6850 & 0.2460 & 0.5639 \\
blend uncalibrated & 0.5741 & 0.6820 & 0.2444 & 0.5642 \\
blend platt calibrated & 0.5546 & 0.6843 & 0.2456 & 0.5545 \\
blend isotonic calibrated & 0.5903 & 0.7392 & 0.2473 & 0.5504 \\
stack uncalibrated & 0.5546 & 0.6852 & 0.2461 & 0.5335 \\
stack platt calibrated & 0.5546 & 0.6874 & 0.2472 & 0.4939 \\
stack isotonic calibrated & 0.5589 & 0.7435 & 0.2507 & 0.5077 \\
\bottomrule
\end{tabular}

\label{tab:oof}
\end{table}

\begin{figure}[!htbp]
\centering
\includegraphics[width=0.60\textwidth]{figures/fig_blend_weights.png}
\caption{Selected blend allocation across component models.}
\label{fig:blendweights}
\end{figure}

\begin{table}[!htbp]
\centering
\caption{Blend Components and Selection Metadata}
\begin{tabular}{lrrrl}
\toprule
component & weight & cv\_log\_loss & cv\_brier & selected \\
\midrule
elo only & 0.6000 & 0.6820 & 0.2444 & yes \\
elo shrunk home rate & 0.5500 & 0.6850 & 0.2460 & yes \\
logistic regression & 0.0000 & 0.6820 & 0.2444 & no \\
random forest & 0.4000 & 0.6820 & 0.2444 & yes \\
hist gradient boosting & 0.0000 & 0.6820 & 0.2444 & no \\
blend uncalibrated & 1.0000 & 0.6820 & 0.2444 & no \\
blend platt calibrated & 0.0000 & 0.6843 & 0.2456 & no \\
blend isotonic calibrated & 0.0000 & 0.7392 & 0.2473 & no \\
stack uncalibrated & 1.0000 & 0.6852 & 0.2461 & no \\
stack platt calibrated & 0.0000 & 0.6874 & 0.2472 & no \\
stack isotonic calibrated & 0.0000 & 0.7435 & 0.2507 & no \\
recent objective elo only log loss &  & 0.6740 &  & no \\
recent objective elo shrunk log loss &  & 0.6749 &  & yes \\
recent objective blend log loss &  & 0.6764 &  & no \\
recent objective stack log loss &  & 0.6796 &  & no \\
recent objective selected score &  & 0.6800 &  & yes \\
\bottomrule
\end{tabular}

\label{tab:blend}
\end{table}

The uncalibrated blend is best on average OOF log loss (Table~\ref{tab:oof}), but recent-fold diagnostics in Table~\ref{tab:blend} indicate better late-fold behavior from Elo.
This OOF-vs-recent divergence is central: complex sources can improve aggregate CV while still degrading the most recent folds that best proxy deployment conditions.
Quantitatively, the best OOF blend reaches 0.6820 log loss, while the recency objective favors Elo-shrunk at 0.6749 recent-fold log loss.
This is a concrete example of temporal risk asymmetry: minimizing average retrospective error is not equivalent to minimizing near-future deployment error.
\FloatBarrier

\subsection{Untouched Holdout Performance}
\begin{table}[!htbp]
\centering
\caption{Untouched Holdout Metrics}
\begin{tabular}{lrrrrr}
\toprule
model & accuracy & log\_loss & brier & auc & delta\_log\_loss\_vs\_baseline \\
\midrule
baseline home rate & 0.6041 & 0.6759 & 0.2414 & 0.5000 & 0.0000 \\
elo only & 0.5888 & 0.6742 & 0.2406 & 0.5284 & -0.0017 \\
elo shrunk home rate & 0.6041 & 0.6720 & 0.2395 & 0.5284 & -0.0039 \\
logistic regression & 0.5482 & 0.7584 & 0.2707 & 0.5215 & 0.0825 \\
random forest & 0.5381 & 0.6888 & 0.2477 & 0.5196 & 0.0130 \\
hist gradient boosting & 0.5584 & 0.8128 & 0.2840 & 0.5139 & 0.1369 \\
blend uncalibrated & 0.5635 & 0.6766 & 0.2419 & 0.5211 & 0.0007 \\
blend final & 0.5635 & 0.6766 & 0.2419 & 0.5211 & 0.0007 \\
stack uncalibrated & 0.5888 & 0.6756 & 0.2413 & 0.5192 & -0.0003 \\
stack final & 0.5888 & 0.6756 & 0.2413 & 0.5192 & -0.0003 \\
elo shrunk recent objective & 0.6041 & 0.6720 & 0.2395 & 0.5284 & -0.0039 \\
\bottomrule
\end{tabular}

\label{tab:holdout}
\end{table}

The selected production strategy is reported in Table~\ref{tab:summary} and \texttt{run\_metadata.json}.
For this run, the selected strategy improves holdout log loss and Brier score versus the home-rate baseline, with tied accuracy (both 0.6041).
The key point is decision-theoretic: a model can tie 0/1 accuracy yet still produce materially better probabilities by reducing confidence error on the same outcomes \cite{gneiting2007}.
Relative to baseline, the selected model reduces holdout log loss by 0.57\% and Brier by 0.79\%, while preserving AUC gains (0.5284 vs 0.5000).

\begin{figure}[!htbp]
\centering
\includegraphics[width=0.72\textwidth]{figures/fig_holdout_selected_vs_baseline.png}
\caption{Direct comparison of selected strategy and baseline on untouched holdout.}
\label{fig:selected_vs_baseline}
\end{figure}

\begin{figure}[!htbp]
\centering
\begin{subfigure}{0.49\textwidth}
    \includegraphics[width=\textwidth]{figures/fig_holdout_model_comparison.png}
    \caption{Model comparison}
\end{subfigure}
\begin{subfigure}{0.49\textwidth}
    \includegraphics[width=\textwidth]{figures/fig_holdout_reliability_plot.png}
    \caption{Reliability}
\end{subfigure}
\caption{Holdout performance and calibration diagnostics.}
\label{fig:holdout}
\end{figure}

\begin{figure}[!htbp]
\centering
\includegraphics[width=0.55\textwidth]{figures/fig_holdout_confusion_matrix.png}
\caption{Confusion matrix for the selected final model on untouched holdout.}
\label{fig:cm}
\end{figure}
\FloatBarrier

\subsection{Uncertainty and Error Analysis}
\begin{table}[!htbp]
\centering
\caption{Holdout Uncertainty and Calibration Diagnostics}
\begin{tabular}{lr}
\toprule
metric & value \\
\midrule
holdout home win rate & 0.604061 \\
mean predicted prob final & 0.588377 \\
mean predicted prob baseline & 0.556951 \\
ece10 final & 0.049618 \\
ece10 baseline & 0.047110 \\
delta log loss mean bootstrap & -0.003863 \\
delta log loss ci 2 5 & -0.015515 \\
delta log loss ci 97 5 & 0.007883 \\
iid prob selected beats baseline & 0.744550 \\
block prob selected beats baseline & 0.706600 \\
\bottomrule
\end{tabular}

\label{tab:uncertainty}
\end{table}

\begin{figure}[!htbp]
\centering
\includegraphics[width=0.78\textwidth]{figures/fig_holdout_probability_density.png}
\caption{Distribution of final predicted probabilities by true holdout outcome.}
\label{fig:prob_density}
\end{figure}

\begin{table}[!htbp]
\centering
\caption{High-Confidence Holdout Errors (Top 10)}
\begin{tabular}{lllrr}
\toprule
game\_id & home\_team & away\_team & home\_win & prob\_final \\
\midrule
\bottomrule
\end{tabular}

\label{tab:hc_errors}
\end{table}

Bootstrap analysis (Table~\ref{tab:uncertainty}) shows mean delta log loss improvement for the selected model versus baseline, but with a confidence interval crossing zero.
This indicates a favorable but statistically modest edge on one season of data.
The high-confidence error table is empty for this run, indicating no extreme-confidence misses under the configured diagnostic threshold; this is directionally consistent with conservative probabilities concentrated near 0.60--0.66.
Combined with the reliability plot, this suggests the remaining error mass is dominated by medium-confidence upset outcomes rather than catastrophic overconfidence.

\subsection{Estimated Test-Set Performance}
To translate holdout evidence into expected test behavior, we estimate the distribution of delta log loss (selected minus baseline) under two resampling regimes:
IID bootstrap \cite{efron1979bootstrap} and moving-block bootstrap (block size 12) to partially account for temporal dependence \cite{kunsch1989jackknife}.
Let \(\Delta^{*(b)}\) be bootstrap replicate \(b\) of delta log loss. We estimate
\[
\hat{p}_{\text{win}}=\frac{1}{B}\sum_{b=1}^{B}\mathbf{1}\{\Delta^{*(b)}<0\},
\]
the probability that selected beats baseline under the chosen resampling regime.

\begin{table}[!htbp]
\centering
\caption{Projected Test-Set Performance from Holdout Resampling}
\begin{tabular}{lr}
\toprule
metric & value \\
\midrule
holdout selected log loss & 0.671998 \\
holdout baseline log loss & 0.675868 \\
iid delta log loss mean & -0.003877 \\
iid delta log loss ci 2 5 & -0.015480 \\
iid delta log loss ci 97 5 & 0.007907 \\
iid prob selected beats baseline & 0.744550 \\
block delta log loss mean & -0.004070 \\
block delta log loss ci 2 5 & -0.018613 \\
block delta log loss ci 97 5 & 0.010200 \\
block prob selected beats baseline & 0.706600 \\
projected selected log loss mean assuming baseline stable & 0.671991 \\
projected selected log loss pessimistic95 assuming baseline stable & 0.683775 \\
projected selected log loss optimistic95 assuming baseline stable & 0.660387 \\
\bottomrule
\end{tabular}

\label{tab:projected}
\end{table}

\begin{figure}[!htbp]
\centering
\includegraphics[width=0.82\textwidth]{figures/fig_delta_bootstrap_distribution.png}
\caption{Estimated distribution of test delta log loss (selected minus baseline) from IID and block bootstrap. Values below zero indicate selected-model improvement.}
\label{fig:delta_bootstrap}
\end{figure}

Table~\ref{tab:projected} provides point and interval projections.
In this run, IID bootstrap estimates a 74.5\% probability that the selected model beats baseline log loss, while block bootstrap estimates 70.7\%.
Under a baseline-stability assumption, the projected selected log loss has mean 0.6720 with an IID 95\% range of approximately [0.6604, 0.6838].
Because both intervals include weak- or no-gain regions, we interpret expected out-of-sample advantage as plausible but modest rather than guaranteed.
This is precisely the profile expected from a one-season setting: positive expected edge with non-trivial overlap between model and baseline risk distributions.
\FloatBarrier

\section{Interpretability and Ranking Outputs}
\begin{figure}[!htbp]
\centering
\begin{subfigure}{0.49\textwidth}
    \includegraphics[width=\textwidth]{figures/fig_feature_importance_logreg.png}
    \caption{Logistic coefficient magnitudes}
\end{subfigure}
\begin{subfigure}{0.49\textwidth}
    \includegraphics[width=\textwidth]{figures/fig_feature_importance_rf.png}
    \caption{Random forest feature importances}
\end{subfigure}
\caption{Feature importance views across model families.}
\label{fig:importance}
\end{figure}

\begin{figure}[!htbp]
\centering
\includegraphics[width=0.70\textwidth]{figures/fig_top_power_rankings.png}
\caption{Top teams by composite power score.}
\label{fig:powervis}
\end{figure}

\begin{table}[!htbp]
\centering
\caption{Top 10 Team Power Rankings}
\begin{tabular}{rlrrr}
\toprule
rank & team & power\_score & elo\_rating & model\_strength \\
\midrule
1 & brazil & 1.0000 & 1563.4111 & 0.5290 \\
2 & netherlands & 0.8713 & 1548.4446 & 0.5222 \\
3 & peru & 0.8101 & 1538.2736 & 0.5224 \\
4 & thailand & 0.7406 & 1535.1534 & 0.5132 \\
5 & india & 0.7142 & 1526.4816 & 0.5181 \\
6 & pakistan & 0.7121 & 1528.2608 & 0.5157 \\
7 & china & 0.6440 & 1523.9226 & 0.5082 \\
8 & panama & 0.6223 & 1511.3110 & 0.5184 \\
9 & iceland & 0.6089 & 1522.7048 & 0.5031 \\
10 & philippines & 0.5171 & 1507.4412 & 0.5034 \\
\bottomrule
\end{tabular}

\label{tab:rankings}
\end{table}
The ranking view is descriptive rather than causal: it summarizes posterior team strength under this model family and season, and should be interpreted jointly with uncertainty diagnostics rather than as a standalone estimator of latent quality.
\FloatBarrier

\section{Round 1 Forecasts}
\begin{figure}[!htbp]
\centering
\includegraphics[width=0.95\textwidth]{figures/fig_round1_probabilities.png}
\caption{Predicted home-win probabilities for Round 1 matchups.}
\label{fig:round1}
\end{figure}

\begin{table}[!htbp]
\centering
\caption{Top Round 1 Home Favorites}
\begin{tabular}{lllr}
\toprule
game\_id & home\_team & away\_team & home\_win\_prob \\
\midrule
game\_1 & brazil & kazakhstan & 0.6622 \\
game\_2 & netherlands & mongolia & 0.6592 \\
game\_3 & peru & rwanda & 0.6481 \\
game\_4 & thailand & oman & 0.6311 \\
game\_6 & india & usa & 0.6286 \\
game\_5 & pakistan & germany & 0.6252 \\
game\_9 & china & france & 0.6218 \\
game\_8 & iceland & canada & 0.6203 \\
\bottomrule
\end{tabular}

\label{tab:round1table}
\end{table}
Round 1 probabilities are intentionally moderate (top forecasts around 0.66), reflecting the model's calibrated stance that even stronger teams retain substantial upset risk in this synthetic environment.
\FloatBarrier

\section{Reproducibility}
Reproducibility is a first-class deliverable in this project.
Each run emits:
\begin{itemize}
    \item \texttt{run\_metadata.json} for model configuration and selected strategy,
    \item \texttt{run\_manifest.json} with command, environment versions, and SHA-256 hashes,
    \item \texttt{visual\_manifest.json} with hashes for all figure files.
\end{itemize}
An automated runner (\texttt{run\_all.sh --verify-repro}) reruns the full pipeline and asserts hash equality for key artifacts.

\section{Limitations}
Several limitations remain.
First, only one season is available, which limits the stability of broad model families.
Second, the competition's synthetic environment may not capture all real hockey non-stationarities.
Third, holdout gains are small in absolute terms and should be interpreted with uncertainty.
Fourth, model calibration can drift under future distribution shift and should be rechecked if new seasons are introduced.
Fifth, we do not run formal pairwise forecast-comparison tests; adding these would further strengthen statistical claims beyond bootstrap interval evidence.

\section{Conclusion}
We present a leak-safe forecasting pipeline aligned to the actual pregame prediction task.
The final system balances interpretability, calibration discipline, and operational reproducibility.
Most importantly, this work shows that temporal correctness, proper scoring, and uncertainty-aware reporting are prerequisites for credible sports analytics claims in competition settings.
In this dataset, conservative probabilistic discipline outperforms higher-variance complexity, yielding a small but repeatable expected log-loss edge against baseline.

\bibliographystyle{plain}
\bibliography{refs}

\end{document}
